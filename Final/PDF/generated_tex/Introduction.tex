\chapter*{Introduction}\label{introduction}

In about 17 years ago writing an operating system's kernel was kind of a
dream for me. Before 2 years of that time I just started my journey with
the wonderful world of computer science through learning programming for
web which made me curios about the different aspects of computers and of
course one of the most interesting of those aspects is operating
systems. At that time I wasn't technically ready yet to write an
operating system kernel, so, a number of experiments to achieve that
goal failed. After these trials, many years passed, I learned a lot
through these years and tackled a number of other system software (such
as compilers, virtual machines and assemblers) to learn how they work
and even implemented some too simple versions of them to make sure that
I've understood their concepts. In 2017 I asked myself, why don't I
implement a simple operating system kernel and achieve one of the oldest
thing in my to-do list which was a kind of dream for me? ``Fine, but how
to make it a useful project for people?'' that's what I told myself as a
response. Going through this journey was interesting for me to learn
more, but I also wanted to make something that's useful for someone
other than me, and at that moment the idea of this book was born. At
that time, I was working on my Master's degree, so I didn't have enough
time to work on this project and that's made me to defer the work on it
until the late of 2019 and after a lot of torture (sorry! dedication?)
this book is finally here.

In this book we are going to start a journey of creating a kernel I
called 539kernel which is a really simple x86 32-bit operating system
kernel that supports multitasking, paging and has its own filesystem. I
wrote 539kernel for this book and made it as simple as possible, so,
anyone would like to learn about operating system kernels can use
539kernel to start. Due to that, some of you may notice that some part
of 539kernel code is written in a naive way, while writing 539kernel I
focused on the readability and easiness of the code instead of the
efficiency. Through this journey your are going to learn a lot about the
basics of operating systems, their kernels and of course the platform
that is going to run 539kernel that we will create together, I mean by
the platform the processors that use x86 architecture. For those who
don't know, an operating system kernel is the core of any operating
system and its job is managing the computer hardware and resources,
distribute these resources for the running programs and provide many
services for those programs to make it easy for them the work with these
resources and hardware.

This book requires a knowledge in \lstinline!C! programming language,
you know; the basics, its syntax, defining variable and functions,
pointers and so on, you don't need to be a master on \lstinline!C!'s
libraries for example. The compiler used to create and test 539kernel is
GNU GCC \lstinline!7.5!. Also, assembly programming language will be
used, but the book doesn't require a knowledge in this language, every
aspect you need to learn about x86 assembly in order to create 539kernel
will be explained in this book. We will use \lstinline!NASM! assembler
for our assembly code and we will use GNU Make to build our kernel,
also, QEMU or Bochs will be used as an emulator to test our work through
this journey. All these three tools will be discussed in chapter
\ref{ch-bootloader} but you need to set them up in your machine. The
full source code of 539kernel is available in GitHub
(\url{https://github.com/MaaSTaaR/539kernel}), there are two directories
in the root directory, \lstinline!src/! is one that contains the last
version of 539kernel, that is, when you finish this book, the code that
you will get will be same as the one in \lstinline!src/!. The directory
\lstinline!evolution_by_versions/! contains the version of 539kernel
while it's under development through the different chapters in this
book. Finally, I hope that you enjoy reading this book and I would be
more than happy to hear your feedback and to help me in spreading this
book which is available freely in (\url{http://539kernel.com}).

\section*{Acknowledgment}\label{acknowledgment}

I would like to thank Dr.~Hussain Almohri \footnote{\url{https://almohri.io/}}
for his kind acceptance to read this book before its release and for his
encouragement, feedback and discussions that really helped me. Also, I
would like to thank my friends Anas Nayfah, Ahmad Yassen, DJ., Naser
Alajmi and my dearest niece Eylaf for their kind support.

\begin{center}\rule{0.5\linewidth}{\linethickness}\end{center}

Mohammed Q. Hussain
(\href{mailto:mqh@539kernel.com}{\nolinkurl{mqh@539kernel.com}})

16 November 2022

Kuwait
