\chapter{Chapter 7: What's Next?}\label{ch-wthat-is-next}

\section{Introduction}\label{introduction}

And now, after writing a simple operating system's kernel and learning
the basics of creating kernels, the question is ``What's Next?''.
Obviously, there is a lot to do after creating 539kernel and the most
straightforward answers for our question are the basic well-known
answers, such as: implementing virtual memory, enabling user-space
environment, providing graphical user interface or porting the kernel to
another architecture (e.g.~ARM architecture). This list is a short list
of what you can do next with your kernel.

Previously, I've introduced the term \emph{kernelist} \footnote{In
  chapter \ref{ch-progenitor} where the distinction between a kernelist
  and a traditionalist has been established.} in which I mean the person
who works on designing operating system kernels with modern innovative
solutions to solve real-world problem. You can continue with your hobby
kernel and implement the well-known concepts of traditional operating
systems that we have just mentioned a little of them, but if you want to
create something that can be more useful and special than a traditional
kernel, then I think you should consider playing the role of a
kernelist.

If you take a quick look on current hobby or even production operating
system kernels through GitHub for example, you will find most of them
are traditional, that is, they focus on implementing the traditional
ideas that are well-known in operating systems world, some of those
kernels go further and try to emulate another previous operating system,
for example, many of them are Unix-like kernel, that is, they try to
emulate Unix. Another examples are ReactOS\footnote{\url{https://reactos.org/}}
which tries to emulate Microsoft Windows and Haiku\footnote{\url{https://www.haiku-os.org/}}
which tries to emulate BeOS which is a discontinued proprietary
operating system. Trying to emulate another operating systems is good
and has advantages of course, but what I'm trying to say that there are
a lot of projects that focus on this line of operating systems
development, that is, the traditionalists line and I think the line of
kernelists needs to be focused on in order to produce more innovate
operating systems.

I've already said that the kernelist doesn't need to propose her own
solutions for the problems that she would like to solve. Instead of
using the old well-known solutions, a kernelist searches for other
better solutions for the given problem and designs an operating system
kernel that uses these solutions. Scientific papers (papers for short)
are the best place to find novel and innovative ideas that solve
real-world problem, most probably, these ideas haven't been implemented
or adopted by others yet\footnote{Scientific papers can be searched for
  through a dedicated search engine, for example, Google Scholar.}.

In this chapter, I've chosen a bunch of scientific papers that propose
new solutions for real-world problem and I'll show you a high-level
overview of these solutions and my goal is to encourage interested
people to start looking to the scientific papers and implement their
solutions to be used in the real-world. Also, I would like to show how
the researches on operating systems field (or simply the kernelists!)
innovate clever solutions and get over the challenges, this could help
an interested person in learning how to overcome his own challenges and
propose innovate solutions for the problem that he faces.

Of course, the ideas on the papers that we are going to discuss (or even
the other operating system's papers) may need more than a simple kernel
such as 539kernel to be implemented. For example, some ideas may need a
networking stack being available in the kernel, which is not available
in 539kernel, so, there will be two options in this case, either you
implement the networking stack in your kernel or you can simply focus on
the problem and solution that the paper present and use an already exist
operating system kernel which has the required feature to develop the
solution upon this chosen kernel, of course, there are many open source
options and one of them is HelenOS\footnote{\url{http://www.helenos.org/}}
microkernel\footnote{The concept of \emph{microkernel} will be explained
  in this chapter.}.

A small note should be mentioned, this chapter only shows an overview of
each paper which means if you are really interesting on the problem and
the solution that a given paper represents, then it's better to read it.
It is easy to get a copy of any mentioned paper in this chapter, you
just need to search for its title in Google Scholar
(\url{https://scholar.google.com/}) and a link to a PDF will show for
you. However, before getting started in discussing the chosen papers, I
would like in the next subsection to discuss a topic that I've deferred
till this point, this topic is related to the architecture design of a
kernel.

\subsection{The Design of Kernel's Architecture: Monolithic
vs.~Microkernel}\label{the-design-of-kernels-architecture-monolithic-vs.microkernel}

The architecture of an operating system, as in any other software, can
be designed in many different ways and the most commonly known kernel's
architecture are \emph{monolithic} and \emph{microkernel}. When the
monolithic architecture is used in a kernel, the whole code of the
kernel runs in the kernel-mode, that is, all the code of the kernel
(hence, all its different modules) has the same privileges to perform
any operation and to change anything in the system, even the device
drivers. A notable example of monolithic kernels is Linux, also, the
modern BSD family (FreeBSD, NetBSD and OpenBSD) uses the monolithic
architecture, the original Unix itself used a monolithic kernel. As you
may notice, 539kernel is also a monolithic kernel.

The other well-known design is microkernel, where not every component of
an operating system kernel is run as a privileged code (in kernel-mode),
instead, only the code that really needs to perform privileged
instructions. The other parts of the kernel that doesn't need to perform
privileged instructions are separated from the microkernel and run in
the user-mode as any other user application and they are known as
\emph{servers} in microkernel architecture. The goal of those servers is
providing an interface that represents the kernel services for the user
applications, and when a privileged instructions needed to be performed,
the server communicates with the microkernel which runs in kernel-mode
to do so. For example, when a user process needs to create a new
process, it needs to communicate with \emph{processes server} which runs
in user-mode and request from it to create a new process. The processes
server maintains the processes list and their process control block
since these data structures don't need to be in the kernel-mode or
kernel's address space, so, when a process creation request arrives, the
creation of the new entry for the new process will be the responsibility
of the processes server with no need to run any privileged code, once a
privileged code is needed, the microkernel will be called by the server.

Let's take process management module of 539kernel as example. This
module provides one function which is \lstinline!process_create! in
\lstinline!process.c!, if you read the code of this module you will see
there is no any part of it needs to run in kernel-mode. That means, if
539kernel was a microkernel, this whole module can run as a userspace
server instead of being in the kernel itself. Another example from
539kernel is the scheduler, you can see for example in the function
\lstinline!scheduler! (\lstinline!scheduler.c!) that there is no need to
run it in the kernel-mode, so, if 539kernel was a microkernel, this
module can be run as a separated userspace server instead. If you review
the code of \lstinline!scheduler! carefully, you can see that it needs
to reach the processes list and also needs to modify the processes
control block, that's mean the scheduler server needs to communicate
with processes server to perform these operations. In microkernels this
can be done through message passing, for example, the scheduler server
can send a message to the processes server to get the ready processes
list or to change some attribute in a process control block and so on,
the same is applicable between the other servers. The function
\lstinline!run_next_process! in \lstinline!scheduler.c! is an example
from 539kernel's scheduler module that needs to run as privileged code,
so, if 539kernel was a microkernel this function should reside in the
kernel itself and not in the scheduler server. Another example of
539kernel that should be in the kernel itself instead of the server is
the interrupt handler \lstinline!isr_32! in \lstinline!idt.asm!.

The goal of microkernel design is keeping the code that needs to run as
privileged code as small as possible and move all the other code to the
userspace. This can make the kernel itself more secure, reliable and
easier to debug. Microkernels have a long history of research to improve
its performance and make it better, there are many microkernels
available nowadays, for example, L4, Mach which has been used in
NeXTSTEP operating system that the current macOS based on \footnote{Though,
  macOS' kernel is considered as a hybrid kernel and not a microkernel.},
Minix, HelenOS and Zircon which is the kernel of Fuchsia operating
system and maybe one of the famous microkernel's related stuff is a
debate known as \emph{Tanenbaum--Torvalds debate} between Andrew S.
Tanenbaum (the creator of Minix and the author of the book ``Operating
Systems: Design and Implementation'') and Linus Torvalds (the creator of
Linux) in 1992 after few months Linux kernel release\footnote{The title
  of the post which started the debate was ``LINUX is obsolete'' by
  Andrew Tanenbaum. The text of the debate is available online here:
  \url{https://www.oreilly.com/openbook/opensources/book/appa.html}}.

\section{In-Process Isolation}\label{in-process-isolation}

In current operating systems, any part of a process can read from and
write to any place of the same process' memory. Consider a web browser
which like any other application consists of a number of different
modules (in the perspective of programmers) and each one of them handles
different functionality, rendering engine is one example of web
browser's module which is responsible for parsing HTML and drawing the
components of the page in front of the user. When an application is
represented as a process, there will be no such distinction in the
kernel's perspective, all application's modules are considered as one
code that each part of it has the permission to do anything that any
other code of the same process can do.

For example, in web browser, the module that stores the list of web
pages that you are visiting right now is able to access the data that is
stored by the module which handles your credit card number when you
issue an online payment. As you can see, the first module is much less
critical than the second one and unfortunately if an attacker can
somehow hack the first module through an exploitable security bug, she
will be able to read the data of the second module, that is, your credit
card information and nothing is going to stop her.

This happens due to the lack of \emph{in-process isolation} in the
current operating systems, that is, both sensitive and insensitive data
of the same process are stored in the same address space and any part of
the process code is permitted to access all these data, so, there is no
difference in your web browser's process between the memory region which
stores that titles of the pages and the region which stores you credit
card information. A severe security bug known as \emph{HeartBleed
vulnerability} showed up due to the lack of in-process isolation. Next,
two of the solutions for the problem of data isolation that has been
proposed by kernelists will be discussed.

\subsection{Lord of x86 Rings}\label{lord-of-x86-rings}

A paper named ``Lord of the x86 Rings: A Portable User Mode Privilege
Separation Architecture on x86'' \footnote{Authored by Hojoon Lee,
  Chihyun Song and Brent Byunghoon Kang. Published on 2018.} proposes an
architecture (named LOTRx86 for short) which provides an in-process
isolation, the paper uses the term \emph{user-mode privilege separation}
which has the same meaning. LOTRx86 doesn't use the new features of the
modern processors to implement the in-process isolation, Intel's
Software Guard Extensions (SGX) is an example of these features. The
reason of not using such modern feature in LOTRx86 is portability, while
SGX is supported in Intel's processors, it is not in AMD's
processors\footnote{Beside Intel, also AMD provides processors that use
  x86 architecture.} which means that employing this feature will make
our kernel only works on Intel's processor and not AMD's. Beside that,
SGX is a relatively new technology\footnote{Intel's SGX is deprecated in
  Intel Core but still available on Intel Xeon.} which means even older
Intel's processors don't support it and that makes our kernel less
portable and can only run on specific types of Intel's processors. So,
if we would like to provide in-process isolation in our kernel, but at
the same time, we want it to work on both Intel's and AMD's processors,
that is, portable \footnote{In LOTRx86 when the term \emph{portable} is
  used to describe something it means that this thing is able to work on
  any modern x86 processor. The same term has another boarder meaning,
  for example, if we use the boarder meaning to say ``Linux kernel is
  \emph{portable}'' we mean that it works on multiple processors
  architecture such as x86, ARM and a lot more and not only on Intel's
  or AMD's x86.}, what should we do? According to LOTRx86, we use
privilege levels to do that.

Throughout this book, we have encountered x86 privilege levels and we
know from our previous discussions that modern operating systems only
use the most privileged level \lstinline!0! as kernel-mode and the least
privileged level \lstinline!3! as user-mode. In LOTRx86 a new area in
each process called \emph{PrivUser} is introduced, this area keeps the
sensitive data of the process and it's only accessible through special
code that runs on the privilege level \lstinline!2!, so, in a kernel
which employs LOTRx86 a process may run in privilege level \lstinline!3!
(user-mode), as in modern operating systems, and may run in privilege
level \lstinline!2! (PrivUser). Most of the normal work of a process
will be done in level \lstinline!3!, but when the code is related to
sensitive data, such as storing, accessing or processing them, the
process will run on level \lstinline!2!. Of course, the sensitive data
cannot be accessed by process' normal code since the latter runs on
level \lstinline!3! and the former needs a code that runs on privilege
level \lstinline!2! to be accessed. If an attacker exploit a
vulnerability that allows him to read the memory of the process, he will
not be able to read the secret data if this vulnerability is on the
normal code of the process.

A kernel with LOTRx86 should provide a way for the programmers to use
the feature provided by LOTRx86, so, the authors of the paper propose a
programming interface named \emph{privcall} which works like Linux
kernel's system calls. Through this interface an application programmer
can write functions (routines) that process the secret data, these
functions will run on privilege level \lstinline!2! and will be stored
in PrivUser, we will call these functions as \emph{secret functions} in
our coming discussion. When the normal code of the process need to do
something with some secret data that is stored in PrivUser a specific
secret function can be called through \lstinline!privcall! interface,
once this call is issued, the current privilege level will be changed
from \lstinline!3! (user-mode) to \lstinline!2! (PrivUser\footnote{In
  the paper, the name PrivUser means two things, the execution mode and
  the secret memory area.}) by using x86 call gates. Note that this
solution \textbf{mitigates} vulnerabilities like HeartBleed but doesn't
\textbf{prevent} them necessarily.

To implement this architecture, two requirements should be satisfied in
order to reach the goal. The first requirement is called
\lstinline!M-SR1! in the paper and it states that the PrivUser area
should be protected from the normal user mode which most of the
application's code run on. The second requirement is called
\lstinline!M-SR2! in the paper and it states that the kernel should be
protected from PrivUser code.

To satisfy the first requirement, the pages of PrivUser are marked as
privileged pages in their page entry \footnote{We have discussed this
  bit in a page entry in chapter \ref{ch-process-management}.}, that is,
the code that run on privilege level \lstinline!3! cannot access them
while the code that runs on levels \lstinline!0!, \lstinline!1! and
\lstinline!2! can. To satisfy the second requirement, the authors
propose to use segmentation, \lstinline!LDT! table is employed to
divided each process into segments and a special segment for the secret
functions and data, that is, PrivUser is defined and the definition of
this segment indicates that the secret functions can only access the
secret data under privilege level \lstinline!2! in order to protect the
kernel's data which reside in privilege level \lstinline!0!.

This was the high-level description of LOTRx86 solution, there are some
challenges that have been faced by the authors and the details of them
and how they overcame them can be found in the paper, so, if you are
interested on implementing LOTRx86 in your kernel, I encourage you to
read the original paper which also discusses how the authors managed to
implement their solution in Linux kernel as kernel modules, also, the
paper shows the performance evaluation of their implementation. There is
something to note, the authors assume that the solution is implemented
in \lstinline!64-bit! environment instead of \lstinline!32-bit! and due
to that they faced some challenges that the may not be faced in
\lstinline!32-bit! environment.

Of course LOTRx86 is not the only proposed solution for our problem,
there are a bunch more and some of them are mentioned on the same paper
that we are discussing. What makes LOTRx86 differs from them is the
focus on a solution that has a better performance and portable as we
have examined in the beginning of this subsection.

As you saw in this solution how the authors played the role of a
kernelist, they proposed a solution for real-world problem, they used
some hardware feature that is usually used in a different way in the
traditional operating systems (privilege level \lstinline!2!) and they
proposed a different and useful idea for operating system kernels.

\subsection{Endokernel}\label{endokernel}

The proposed solution In LOTRx86 paper isolates the memory within the
process but what about the other system resources (e.g.~files)? For
example, what if a critical module in the process needs to read and
write on a secret file while the other modules of the same process
should not reach this file at all. The only system resource that LOTRx86
protects is the memory while the other resources of the system are
accessible by any module within the process.

The paper ``The Endokernel: Fast, Secure, and Programmable Subprocess
Virtualization'' \footnote{Authored by: Bumjin Im, Fangfei Yang,
  Chia-Che Tsai, Michael LeMay, Anjo Vahldiek-Oberwagner and Nathan
  Dautenhahn. Published on 2021.} proposes a solution to handle this
case by modifying the traditional process model which used by most
modern operating systems. In Endokernel Architecture a monitor is
attached within each process. This monitor, which is called endokernel,
isolates itself from the untrusted parts of the process and also
provides a lightweight virtual machine, called endoprocess, to the rest
of the process and through defined polices the isolation can be
enforced, for example, some processor's instructions can be permitted to
be executed by the untrusted parts of the process without monitoring but
some other can be defined that the should be monitored. Also, the
filesystem's operations that are allowed to be used can be defined by
the policies and the endokernel is going to ensure that these policies
are enforced.

\section{Nested Kernel}\label{nested-kernel}

In monolithic design, the kernel is considered as one entity and each
component of the kernel is able to read/modify the data and maybe the
code of another component since the whole of the kernel's code works on
kernel mode. Beside the standard components of the kernel (e.g.~process
management and memory management), usually, the device drivers are
considered as a part of the monolithic kernel and they run on the kernel
mode, these device drivers are, most probably, written by a third party
entity which makes them considered as an untrusted code, also, they may
be buggy if they are compared to the standard code of the kernel. Any
exploitable bug in any part of a monolithic kernel (either in a device
driver or not) will give the attacker the access to the whole kernel.
This problem reminds us with the problem which has been presented
earlier in this chapter but this time the kernel is the one which
suffers from it.

Microkernel design solves this problem by separating the most components
of the kernel as user-space servers, but what if we would like to keep
the monolithic design and have this kind of separation? This is what a
paper titled ``Nested Kernel: An Operating System Architecture for
Intra-Kernel Privilege Separation'' \footnote{Authored by Nathan
  Dautenhahn, Theodoros Kasampalis, Will Dietz, John Criswell and Vikram
  Adve. Published on 2015.} is trying to do by proposing a new kernel's
design called \emph{nested kernel}.

Memory is the root of all evil, that's what I feel this paper is trying
to tell us. In nested kernel design, the operating system kernel is
divided into two parts, the first one is nested kernel and the second
part is \emph{outer kernel}. The nested kernel is isolated from the
outer kernel and both parts run on kernel mode. The job of nested kernel
is to isolate the memory management unit (MMU) from the outer kernel. To
make the outer kernel able to use the functionality that MMU provides,
the nested kernel exposes and controls an interface of the MMU, this
interface is called \emph{virtual MMU} (\emph{vMMU}) in the paper, so,
if any part of the outer kernel needs to manipulate the state of MMU
then vMMU interface can be used. The nested kernel part has small and
trusted code while the outer kernel contains all other code that cannot
be trusted (e.g.~device drivers) or may be buggy. When we say isolating
MMU we mean that the data structures and registers that build the state
of MMU are isolated, so, in x86 for example, isolating MMU means
isolating page directory, page tables and the control registers that are
related to paging.

The memory regions which a kernel implementer would like to protect from
being modified by the outer kernel (protected memory) are marked as
read-only region in nested kernel design and only the nested kernel has
the permission to modify them. For example, say that you have decided to
protect the memory that contains the code of the kernel which checks if
the current user has the permissions to read or modify a specific file,
this region can by marked as read-only and can be protected by the
nested kernel all the time from being modified by any part of the outer
kernel. Now, assume that an attacker found an exploitable security bug
in one of the device drivers, and his goal is to modify that code of
permission checking in order to let him to read some critical file, this
cannot be done since the memory region is protected and read-only, the
paper discusses how in details how to ensure that the outer kernel
doesn't violate the protection of nested kernel in x86 architecture.

That's not the whole story. Making the nested kernel the only way to
modify the protected memory by the outer kernel means that the nested
kernel can be a mediator which will be called before any modification
performed. This will let the kernel's implementer to define security
policies and enforce them while the system is running. For example, the
authors propose \emph{no write policy} which doesn't let the outer
kernel to write on a specific memory region at all (e.g.~the example of
checking permissions code). Another proposed policy is \emph{write-once
policy} which lets the outer kernel to write to a region of memory just
one time, this policy will be useful with the memory region that
contains the \lstinline!IDT! table for example, so, the attacker cannot
modify the interrupt service routines after setting them up by the
trusted code of outer kernel. More policies were presented in the paper.
You can see here how the kernelists proposed a new kernel design other
than the popular ones (microkernel and monolithic) in order to solve a
specific real-world problem.

\section{Multikernel}\label{multikernel}

The paper ``The Multikernel: A new OS architecture for scalable
multicore systems'' \footnote{Authored By: Andrew Baumann, Paul Barham,
  Pierre-Evariste Dagand, Tim Harris, Rebecca Isaacs, Simon Peter,
  Timothy Roscoe, Adrian Schüpbach and Akhilesh Singhania. Published in
  2009.} shows a good example of kernelists who get rid of the old
designs completely in order to provide a modern one which is more
suitable for current days. In the paper, the authors have observed the
new trends in the modern hardware, these trends motivated them to
propose a new kernel architecture named \emph{multikernel}. One of these
observations is the diversity of the new systems, according to the
authors, the operating systems in the new systems need to work with
machines that may have cores with different instruction set
architectures, that is, they have heterogeneous cores, either in term of
instruction set architecture or performance. Another observation is that
the message passing is now easier in the modern hardware and can be used
instead of shared memory in order to share information between two
processes for example, the idea of multikernel aims to handle these
observations and provide an architecture of a kernel that is suitable
for the modern multicore systems.

In multikernel architecture, a multicore system is handled as a network
of cores, as if each core is a separate processor, and the
communications between the cores are performed through message passing,
it is not necessary that the cores belong to the same machine. When the
cores are handled as a network of machines, the algorithms and
techniques of distributed systems can be used.

The design of multikernel depends on three principles. First, all
communications between the cores in the kernel should be explicit
through message passing and no implicit communications (e.g.~through
shared memory) is allowed, one of the benefits of this principle is the
ability to use well-known networking optimizations in order to make the
communications between cores more efficient. Also, making the
communication explicit can help in reasoning about the correctness of
the kernel's code. The second principle is separating the structure of
the operating system as much as possible from the hardware, that is, the
structure should be hardware neutral. The benefits of this principle are
obvious and one of them is making the adaptation of processor's specific
optimization easier \footnote{The paper mentioned that applying one of
  optimizations on Windows 7 caused changes in \lstinline!6,000! lines
  of code through \lstinline!58! files.}. The last principle is dealing
with the state of the operating system (e.g.~processes table) as
replicated instead of shared, that is, when a core need to deal with a
global data structure, a copy of this data structure is sent to this
core instead of using just one copy by all the cores in the system.
Based on these design principles, the authors built an implementation
called Barrelfish, according to the authors, this implementation is an
example of multikernel but not the only way to build one. The paper
discusses in details how they designed Barrelfish to realize
multikernel's design principles and how they overcame the challenges
that the have faced.

\section{Dynamic Reconfiguration}\label{dynamic-reconfiguration}

Changing a specific module while the system is running can be an
important aspect in some systems, for most desktop users, when some
module of a system is changed, due to updating the system for example,
it will be fine to reboot the system to get the new changes applied, but
what about a server that needs to run all the time with no downtime,
rebooting it is not an option. Current operating systems still require a
reboot when an update to specific parts is performed, for example,
updating Linux kernel in a running system requires a reboot to this
system to be able to use the new version of the kernel.

Dynamic reconfiguration is the process of changing a specific module of
the system while keeping it running without the need of rebooting it,
that's how a paper titled ``Building reconfigurable component-based OS
with THINK'' \footnote{Authored by: Juraj Polakovic, Ali Erdem Özcan and
  Jean-Bernard Stefani. Published on 2006} defines this term. According
to the paper, dynamic reconfiguration consists of the followings steps:
First, the part that we would like to reconfigure (the reconfiguration
target) should be identified and separated from other parts, to do that,
THINK framework uses a component model called Fractal \footnote{\url{https://fractal.ow2.io/}}
in order to identify each part of the system as a separated component,
after that, the process of reconfiguration is going to deal with these
components, for example, in 539kernel the process management part, the
scheduler, the memory management part, the allocator and the filesystem
can be defined as separated components, as you can see each of these
part has its own functionality and can be encapsulated, by using dynamic
reconfiguration we can for example change the current scheduler with
another one while the system is running.

The second step is to make sure that the reconfiguration target is on
the safe state, that is, there is no other part that is using the target
right now, thread counting is one technique that has been proposed in
the paper to detect safe state, when employing this technique any call
to a component causes the thread counter to increase by \lstinline!1!
and when this call finishes the thread counter decreases by
\lstinline!1!, a component is on the safe state when the thread counter
is \lstinline!0!, that is, no thread (or process) is currently using the
target component. After the target component reaches the safe state it
can be changed to the new component, the context of the target should be
moved to the new component and after that the execution of the component
can be resumed.

\section{Unikernel}\label{unikernel}

Virtualization is widely used today and cloud computing is an obvious
example of employing virtualization technologies. Nowadays, you can
easily start and stop virtual machines that run a commodity operating
system (e.g.~Linux or Windows) and via this virtual machine you can run
whatever software you wish as if this virtual machine is a real one.
There are many cases where a virtual machine is used to provide just one
thing, for example, a virtual machine that runs a web server solely. To
do that, of course an operating system is needed to be installed in the
virtual machine, say for example Linux, and of course a web server
should be installed on top this operating system, say Apache. Linux (and
modern general purpose operating systems) is a multiprocess and
multiuser kernel which contains a lot of code that handle the protection
of the processes, the users and the kernel itself (via the separation of
kernel-mode and user-mode as we have discussed through this book),
beside that, there are a lot of services that are provided in general
purpose operating systems so they can be suitable for all users.

In our example of the virtual machine which only runs a web server all
of these services are not needed, they can be omitted and only the
services that are needed by the web server are kept, this is what
\emph{unikernels} do. In this model of kernels design, everything that
is not needed is omitted, even the separation between the kernel and the
user application (in our example the web server) and let both of them to
run in a single address space. All of these changes on the kernel's
architecture provide us with many benefits, the size of the kernel and
the final binary will be smaller, it will have a better performance
\footnote{In the website of a unikernel called Unikraft the following is
  stated: ``On Unikraft, NGINX is 166\% faster than on Linux and 182\%
  faster than on Docker''.}, it will boot faster and the attack surface
will be smaller.

I think unikernel is a good path to start your journey as a kernelist,
especially that this topic is gaining a momentum these days. The idea
behind a unikernel is simple, a skeleton of an operating system's kernel
which targets a specific architecture (e.g.~x86) is provided to the user
with specific services (e.g.~functions and so on) to make it easy for
the user to write his own application, in this stage, the combination of
the kernel and those provided services is known as a \emph{library
operating system}, after writing the application, say a web server, both
the application and the kernel are built as one entity which is the
unikernel that is going to run on a virtual machine and provide a
specific service.

There are many library operating systems available, for example:
IncludeOS \footnote{\url{https://www.includeos.org/}} which its design
is presented in a paper titled ``IncludeOS: A minimal, resource
efficient unikernel for cloud services'' \footnote{Authored by Alfred
  Bratterud, Alf-Andre Walla, Harek Haugerud, Paal E. Engelstad and
  Kyrre Begnum. Published on 2015.}, Unikraft \footnote{\url{https://unikraft.org/}}
which its design is presented in a paper titled ``Unikraft: Fast,
Specialized Unikernels the Easy Way'' \footnote{Authored by Simon
  Kuenzer, Vlad-Andrei Bădoiu, Hugo Lefeuvre, Sharan Santhanam,
  Alexander Jung, Gaulthier Gain, Cyril Soldani, Costin Lupu, Stefan
  Teodorescu, Costi Răducanu, Cristian Banu, Laurent Mathy, Răzvan
  Deaconescu, Costin Raiciu and Felipe Huici. Published on 2021.}, OSv
\footnote{\url{https://osv.io/}} and MirageOS \footnote{\url{https://mirage.io/}}.
Also, there are many new scientific papers that try to find solutions
for unikernel problems and advance the area. For example, the paper
``Towards a Practical Ecosystem of Specialized OS Kernels'' \footnote{Authored
  by Conghao Liu and Kyle C. Hale. Published on 2019.} proposes a way to
build an ecosystem for library operating systems which helps the user to
find a kernel that fits his needs and helps in building the last result
of the user's application. Another paper is titled ``A Binary-Compatible
Unikernel'' \footnote{Authored by Pierre Olivier, Daniel Chiba, Stefan
  Lankes, Changwoo Min and Binoy Ravindran. Published on 2019} which
proposes a unikernel named HermiTux \footnote{\url{https://ssrg-vt.github.io/hermitux/}}
that provides binary compatibility with Linux applications, that is,
instead of writing a wholly new application to be used as a unikernel,
with binary compatibility one of mature applications that already exists
for Linux can be used instead, for example, running Apache web server a
unikernel instead of writing a wholly new web server is most probably
better idea.

Of course, there are a lot more papers either about unikernels or any
other operating system topics, just search for them and you will find a
lot. I hope you a happy kernelist/traditionalist journey and thanks for
reading this book!
